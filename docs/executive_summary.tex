\section*{Executive Summary}

The Causal Resonance Unification (CRU) framework is a new approach to unifying physics based on resonance principles. At its foundation is a modified mass–energy equivalence relation,
\[
m c^2 = \hbar \omega \zeta ,
\]
which identifies every particle mass with a locked resonance frequency. The factor $\zeta$ accounts for interaction dressing and ensures flexibility while remaining tied to measurable data. This simple but powerful relation anchors the Standard Model, gravitation, and cosmology into a unified resonance picture.

Key features of CRU:
\begin{itemize}
  \item \textbf{Data Anchoring:} Reproduces known particle masses (electron, muon, Higgs) as resonance anchors without arbitrary parameters.
  \item \textbf{Mathematical Framework:} Provides a resonance field equation, stress--energy tensor, dispersion relation, and renormalization group flows.
  \item \textbf{Cosmology Compatibility:} Satisfies constraints from big bang nucleosynthesis (BBN), cosmic microwave background (CMB), structure formation, and fifth--force bounds.
  \item \textbf{Falsifiability:} Predicts resonance locking windows that can be probed experimentally; failure to detect signals within defined ranges rules out the theory.
  \item \textbf{Experimental Roadmap:} Outlines tabletop, collider, and astrophysical experiments to verify or falsify CRU predictions.
\end{itemize}

Unlike many unification frameworks, CRU is not philosophical or speculative: it is explicitly falsifiable, experimentally anchored, and numerically testable. This positions it as a clean, transparent alternative to string theory and loop quantum gravity, offering a resonance--based foundation for physics beyond the Standard Model.
